\documentclass[hyperref=unicode, presentation,10pt]{beamer}

\usepackage[absolute,overlay]{textpos}
\usepackage{array}
\usepackage{graphicx}
\usepackage{adjustbox}
\usepackage{mhchem}
\usepackage{chemfig}
\usepackage{ucs}
\usepackage[utf8]{inputenc}
\usepackage{caption}

\PrerenderUnicode{ěščřžýáíéĚŠČŘŽÝÁÍÉďťňĎŤŇůúÚóÓ}

\addtobeamertemplate{frametitle}{
   \let\insertframetitle\insertsectionhead}{}
\addtobeamertemplate{frametitle}{
   \let\insertframesubtitle\insertsubsectionhead}{}

\makeatletter
  \CheckCommand*\beamer@checkframetitle{\@ifnextchar\bgroup\beamer@inlineframetitle{}}
  \renewcommand*\beamer@checkframetitle{\global\let\beamer@frametitle\relax\@ifnextchar\bgroup\beamer@inlineframetitle{}}
\makeatother
\setbeamercolor{section in toc}{fg=red}
\setbeamertemplate{section in toc shaded}[default][100]

\usepackage{fontspec}
\usepackage{unicode-math}

\usepackage{polyglossia}
\setdefaultlanguage{czech}

\def\uv#1{„#1“}

\mode<presentation>{\usetheme{default}}
 \usecolortheme{Whale}

\setbeamertemplate{footline}[frame number]

\begin{document}
\title[Crisis]
{C2062 -- Anorganická chemie II}

\subtitle{Bioanorganická chemie -- kovy v biologických systémech}
\author{Zdeněk Moravec, hugo@chemi.muni.cz \\ \adjincludegraphics[height=50mm]{img/Chlorophyll_b_structure.png}}
\date{}

\begin{frame}
	\titlepage
\end{frame}

\section{Bioanorganická chemie}
\frame{
	\frametitle{}
	\vfill
	\begin{itemize}
		\item Mezioborová vědní disciplína, stojí mezi chemií anorganickou, organickou a biochemií.
		\item Studuje funkci anorganických látek v biologických systémech.
	\end{itemize}
	\begin{figure}
		\adjincludegraphics[height=0.6\textheight]{img/HemerythrinTri.jpg}
		\caption*{Hemerythrin, protein obsahující železo}
	\end{figure}
	\vfill
}

\frame{
	\frametitle{}
	\vfill
	\begin{itemize}
		\item Hlavními oblastmi výzkumu jsou:
		\begin{itemize}
			\item Metalloproteiny, metalloenzymy, biologicky aktivní komplexní sloučeniny.
			\item Transport a uchovávání kovů v organismu.
			\item Biomineralizace -- mechanismus biologického vzniku minerálů.
			\item Kovy v medicíně, např. \textit{cisplatina}.
			\item Toxicita kovů pro člověka a jiné organismy.
			\item Kovy v životním prostředí.
		\end{itemize}
	\end{itemize}
\begin{figure}
	\adjincludegraphics[height=0.4\textheight]{img/Cisplatin-stereo.png}
\end{figure}
	\vfill
}

\frame{
	\frametitle{}
	\vfill
	\begin{itemize}
		\item Kovy v lidském těle (o váze 70 kg)
		\begin{tabular}{|c|c|l|}
			\hline
			\textbf{Kov} & \textbf{Obsah [mg]} & \textbf{Funkce} \\\hline
			V & 0,1 & Enzymy \\\hline
			Co & 3 & Vitamín B12 \\\hline
			Mo & 5 & Enzymy \\\hline
			Mn & 12 & Enzymy; fotoredoxní aktivita ve fotosystému II \\\hline
			Cr & 14 & Metabolismus glukózy \\\hline
			Ni & 15 & Enzymy \\\hline
			Cu & 72 & Přenos a ukládání \ce{O2}; přenos elektronů \\\hline
			Zn & 2300 & Lewisova kyselina \\\hline
			Fe & 4200 & FeS proteiny, přenos a ukládání \ce{O2} a \ce{CO2} \\\hline
			Na & 90 000 & Extracelulární tekutiny \\\hline
			K & 120 000 & Intracelulární tekutiny \\\hline
		\end{tabular}
	\end{itemize}
	\vfill
}

\frame{
	\frametitle{}
	\vfill
	\begin{figure}
		\adjincludegraphics[width=\textwidth]{img/Periodic_table-bio.png}
		\caption*{Žlutá -- makroprvky vyskytující se v organismech; zelená -- stopové prvky; červená -- stopové prvky vyskytující se jen v některých organismech}
	\end{figure}
	\vfill
}

\section{Sodík a draslík}
\frame{
	\frametitle{}
	\vfill
	\begin{itemize}
		\item Oba prvky jsou velmi důležité pro všechny živočichy, vč. člověka.
		\item Sodík je součástí mimobuněčných tekutin.
		\item Draslík je součástí nitrobuněčných tekutin.
		\item Jejich transport skrz buněčnou membránu zajišťuje \textit{sodno-draselná pumpa}.
	\end{itemize}
	\begin{figure}
		\adjincludegraphics[height=.55\textheight]{img/Scheme_sodium-potassium_pump.png}
	\end{figure}
	\vfill
}

\section{Hořčík}
\frame{
	\frametitle{}
	\vfill
	\begin{itemize}
		\item Hořčík je součástí chlorofylu, zeleného rostlinného pigmentu, který se účastní fotosyntézy.
		\item Procesu, kdy z oxidu uhličitého a vody vzniká v přítomnosti světla cukr.
	\end{itemize}
	\begin{figure}
		\adjincludegraphics[height=.6\textheight]{img/Chlorophyll_d_structure.png}
	\end{figure}
	\vfill
}

\frame{
	\frametitle{}
	\vfill
	\begin{figure}
		\adjincludegraphics[height=.8\textheight]{img/Photosynthesis_en.png}
	\end{figure}
	\vfill
}

\section{Vanad}
\frame{
	\frametitle{}
	\vfill
	\begin{columns}
		\begin{column}{.75\textwidth}
			\begin{itemize}
				\item Vanad má důležitější roli v mořském prostředí než v suchozemském.
				\item Mořské řasy produkují vanadovou bromoperoxidasu, chloroperoxidasu a jodoperoxidasu, které jsou odpovědné za odstraňování peroxidu z organismu:
				\item \ce{R-H + Br^- + H2O2 -> R-Br + H2O + OH-}
				\item Muchomůrky červené mají schopnost silně akumulovat vanad z okolí.
				\item Vanad se v nich vyskytuje jako \textit{amavadinový anion}, obsahuje vanad v oxidačním stavu IV, který je chelatován dvěma anionty kyseliny N-hydroxyimino-2,2'-dipropionové.
			\end{itemize}
		\end{column}
		\begin{column}{.4\textwidth}
			\begin{figure}
				\adjincludegraphics[height=.35\textheight]{img/VBrPOactsite.png}
			\end{figure}
			\begin{figure}
				\adjincludegraphics[height=.35\textheight]{img/Amanita_muscaria._Eastern_Siberia.jpg}
			\end{figure}
		\end{column}
	\end{columns}
	\vfill
}

\frame{
	\frametitle{}
	\vfill
	\textbf{Struktura amavadinu}\footnote[frame]{\href{https://doi.org/10.1002/(SICI)1521-3773(19990315)38:6\%3C795::AID-ANIE795\%3E3.0.CO;2-7}{The Structural Characterization of Amavadin}}
	\begin{columns}
		\begin{column}{.5\textwidth}
			\begin{figure}
				\adjincludegraphics[width=\textwidth]{img/Amavadin.png}
			\end{figure}
		\end{column}
		\begin{column}{.5\textwidth}
			\begin{figure}
				\adjincludegraphics[width=\textwidth]{img/Amavadin-from-xtal.png}
			\end{figure}
		\end{column}
	\end{columns}
	\vfill
}

\section{Chrom}

\section{Molybden}
\frame{
	\frametitle{}
	\vfill
	\begin{itemize}
		\item Jeho koncentrace v živých organismech je nízká, ale i tak je nezbytný.
		\item Nedostatek molybdenu u lidí není příliš častý, může způsobit mentální poruchy.\footnote[frame]{\href{https://lpi.oregonstate.edu/mic/minerals/molybdenum}{Molybdenum}}
		\item Nedostatek molybdenu u květáku a brokolice způsobuje tzv. \textit{vyslepnutí}, čímž je myšleno netvoření růžic, příp. tvorba silně redukovaných růžic.\footnote[frame]{\href{http://eagri.cz/public/app/srs_pub/fytoportal/public/?key="c18ccd9cbe2ba381e37b810d0c71a00f"\#rlp|poruchy|detail:c18ccd9cbe2ba381e37b810d0c71a00f|popis}{Mo-deficientní vyslepnutí květáku a brokolice}}
		\item U kukuřice způsobuje nedostatek molybdenu předčasné klíčení semen.\footnote[frame]{\href{https://dx.doi.org/10.1007/978-94-011-3438-5_41}{Soil acidity effects on premature germination in immature maize grain}}
		\item Molybden se účastní fixace dusíku a metabolismu fosforu.
		\item Je součástí bílkoviny \textit{molybdoferredoxinu}, která obsahuje \ce{Fe-S} motiv a molybden oktaedricky koordinovaný sírou.\footnote[frame]{\href{https://www.biologyonline.com/dictionary/molybdoferredoxin}{Molybdoferredoxin}}
	\end{itemize}
	\vfill
}

\frame{
	\frametitle{}
	\vfill

	\begin{figure}
		\adjincludegraphics[height=.8\textheight]{img/Premature_germination_maize_2014_05_15_10_35_37_7919.jpg}
	\end{figure}
	\vfill
}


\section{Wolfram}
\frame{
	\frametitle{}
	\vfill
	\begin{itemize}
		\item Wolfram je nejtěžším kovem, který se vyskytuje v biologických systémech.
		\item Vyskytuje se u některých prokaryotních bakterií, kde je součástí enzymů oxidoreduktas, např. \textit{aldehyd ferredoxin oxidoreduktázy}.\footnote[frame]{\href{https://doi.org/10.1016/S0076-6879(01)31052-2}{Aldehyde Oxidoreductases from \textit{Pyrococcus furiosus}}}
	\end{itemize}

	\begin{figure}
		\adjincludegraphics[height=.5\textheight]{img/AOR_Mechanism.jpg}
	\end{figure}
	\vfill
}

\section{Mangan}
\frame{
	\frametitle{}
	\vfill
	\begin{itemize}
		\item Mangan je pro organismus člověka nezbytný, dlouhodobý nedostatek vede k cévním problémům. Dochází ke změnám metabolismu cholesterolu a jeho ukládání na cévní stěny.
		\item Také má důležitou roli v metabolismu cukrů a jeho nedostatek může způsobit cukrovku.
		\item Nadbytek manganu může vést k problémům v nervové soustavě a dlouhodobě zvýšený příjem může způsobit až Parkinsonovu nemoc.
		\item Je součástí superoxid dismutásy 2 (SOD2).
		\item Doporučená denní dávka pro člověka je 2--5 mg denně. Hlavními zdroji jsou obilniny, hrášek, špenát a ořechy.\footnote[frame]{\href{https://www.bezpecnostpotravin.cz/vyzivovy-vyznam-manganu.aspx}{Výživový význam manganu}}
	\end{itemize}
	\vfill
}

\frame{
	\frametitle{}
	\vfill
	\begin{figure}
		\adjincludegraphics[width=\textwidth]{img/SOD2_Proton.jpg}
		\caption*{Mechanismus přenosu elektronu pomocí SOD2 proteinu}
	\end{figure}
	\vfill
}

\section{Železo}
\frame{
	\frametitle{}
	\vfill
	\begin{itemize}
		\item Železo je asi nejdůležitějším přechodným kovem pro biologii živočichů i rostlin.
		\item Tělo dospělého člověka obsahuje zhruba 4 g železa, z toho 3 gramy připadají na \textit{hemoglobin}.
		\item Hemoglobin je bílkovina transportující kyslík, je obsažen v červených krvinkách.\footnote[frame]{\href{https://www.wikiskripta.eu/w/Transport_kyslíku_krví}{Transport kyslíku krví}}
		\item Obsahuje železnatý ion ve vyskospinovém stavu komplexovaný porfyrinovým ligandem.
		\item Po navázání kyslíku, nedojde k oxidaci na \ce{Fe^{III}}, ale ke změně stavu na nízkospinový, diamagnetický. Zároveň se na železo váže histidin.
		\item Kromě kyslíku, transportuje hemoglobin i \ce{CO2}.
	\end{itemize}
	\begin{columns}
		\begin{column}{.5\textwidth}
			\begin{figure}
				\adjincludegraphics[height=.25\textheight]{img/Heme1.png}
			\end{figure}
		\end{column}
		\begin{column}{.5\textwidth}
			\begin{figure}
				\adjincludegraphics[height=.25\textheight]{img/Redbloodcells.jpg}
			\end{figure}
		\end{column}
	\end{columns}
	\vfill
}

\frame{
	\frametitle{}
	\begin{itemize}
		\item Navázáním CO na hemoglobin vzniká \textit{karboxylhemoglobin}. CO je vázán podstatně pevněji než kyslík, to je důvodem nebezpečnosti oxidu uhelnatého pro člověka.
	\end{itemize}
	\begin{figure}
		\adjincludegraphics[width=1.1\textwidth]{img/Structures_of_Hemoglobin.png}
	\end{figure}
}

\frame{
	\frametitle{}
	\begin{figure}
		\adjincludegraphics[width=0.9\textwidth]{img/Hemoglobin-3D-model-ribbons.png}
		\caption*{3D struktura hemoglobinu}
	\end{figure}
}

\frame{
	\frametitle{}
	\begin{itemize}
		\item Struktura hemoglobinu brání nežádoucí oxidaci \ce{Fe^{2+}} na \ce{Fe^{3+}}.
		\item Okolí hemu vytváří hydrofóbní prostředí, které se vlastnostmi nepolárnímu rozpouštědlu s nízkou permitivitou.
		\item Tím zabraňuje přístupu kyslíku k železu.
	\end{itemize}
	\begin{figure}
		\adjincludegraphics[height=.5\textheight]{img/Heme1.png}
	\end{figure}
}

\frame{
	\frametitle{}
	\vfill
	\begin{itemize}
		\item Železo je součástí i jiných bílkovin, ty často obsahují vazbu \ce{Fe-S} (tzv. FeS proteiny).
		\item Železo je vázáno k postranním řetězcům aminokyselin \textit{cysteinu} a \textit{histidinu}.\footnote[frame]{\href{https://doi.org/10.1016/B978-0-12-378630-2.00222-X}{Iron–Sulfur Proteins}}
		\item Tyto proteiny mají funkci transferu elektronů (oxidoreduktasy nebo transelektronasy).
		\item Během transferu elektronů dochází ke změně oxidačního stavu železa z II na III, oba stavy jsou ve vysokospinové konfiguraci.
	\end{itemize}
	\begin{columns}
		\begin{column}{.5\textwidth}
			\begin{figure}
				\adjincludegraphics[height=.3\textheight]{img/Cystein.png}
				\caption*{Cystein}
			\end{figure}
		\end{column}
		\begin{column}{.5\textwidth}
			\begin{figure}
				\adjincludegraphics[height=.3\textheight]{img/Histidin.png}
				\caption*{Histidin}
			\end{figure}
		\end{column}
	\end{columns}
	\vfill
}

\section{Kobalt}
\frame{
	\frametitle{}
	\vfill
	\begin{itemize}
		\item Kobalt je esenciální pro metabolismus všech živočichů.
		\item Je složkou vitamínu B12, označovaného jako \textit{kobalamin}.
		\item Vitamín byl objeven roku 1926 G. R. Minotem a W. P. Murphym.
		\item Jeho hlavní funkcí je regulace syntézy DNA, ale podílí se také na syntéze mastných kyselin a produkci energie.
		\item Bakterie v žaludku přežvýkavců dokáží zpracovat soli kobaltu na vitamín B12, proto je jeho přítomnost v půdě (v nízké koncentraci) důležitá pro zdraví pasoucích se zvířat.
		\item Na konci 19. století bylo zjištěno, že zhoubné onemocnění ovcí a hovězího dobytka je způsobeno právě nedostatkem kobaltu a nikoliv železa, jak se dříve předpokládalo.
		\item U člověka způsobuje nedostatek vitamínu B12 chudokrevnost, únavu, zácpu, pokles váhy. Může způsobovat i neurologické změny (deprese).
	\end{itemize}
	\vfill
}

\frame{
	\frametitle{}
	\vfill
	\begin{columns}
		\begin{column}{.5\textwidth}
			\begin{figure}
				\adjincludegraphics[height=.8\textheight]{img/Cobalamin.png}
			\end{figure}
		\end{column}
		\begin{column}{.5\textwidth}
			\begin{figure}
				\adjincludegraphics[height=\textwidth,angle=90]{img/Cyanocobalamin.png}
			\end{figure}
		\end{column}
	\end{columns}
	\vfill
}

\frame{
	\frametitle{}
	\vfill
	\begin{columns}
		\begin{column}{.6\textwidth}
			\begin{itemize}
				\item Hlavním zdrojem vitamínu B12 jsou živočišné produkty: maso, vejce, sýry.
				\item Doporučená denní dávka je 2--3~$\mu$g denně.
				\item Kobalamin je oranžová, diamagnetická látka.
				\item Koordinační sféra je obdobná, jako u železa v hemu.
				\item Kobalt je koordinován ke čtyřem dusíkům v rovině korrinového kruhu, pátý dusík je nad rovinou kruhu.
				\item Šestá pozice je obsazena uhlíkovým atomem z ligandu R.
			\end{itemize}
		\end{column}
		\begin{column}{.4\textwidth}
			\begin{figure}
				\adjincludegraphics[height=.8\textheight]{img/B12_1000mcg.jpg}
			\end{figure}
		\end{column}
	\end{columns}
	\vfill
}

\section{Měď}
\frame{
	\frametitle{}
	\vfill
	\begin{itemize}
		\item Měď patří mezi prvků důležité pro živé organismy.
		\item Vyskytuje se v řadě enzymatických cyklů, např. v metabolismu sachardidů a také při tvorbě kostní hmoty a červených krvinek.
		\item Měď je součástí \textit{hemocynianu}, analogu hemoglobinu u některých živočichů.
		\item Denní dávka mědi by se měla pohybovat mezi 1 a 100~mg. Zdroji mědi jsou ořechy, houby, korýši, měkkýši, játra a kakao.
		\item Nedostatek mědi se projevuje chudokrevností, zhoršením metabolismu sacharidů a zpomalením duševního vývoje.
		\item Při předávkování mědí hrozí podobné obtíže jako u kadmia a rtuti.
	\end{itemize}
	\vfill
}

\frame{
	\frametitle{}
	\vfill
	\begin{columns}
		\begin{column}{.6\textwidth}
		\begin{itemize}
		\item Hemocyanin, je metaloprotein obsahující dva ionty mědi.
		\item Je součástí respiračního cyklu měkkýšů a některých členovců.
		\item Ionty mědi slouží k navázání molekuly kyslíku.
		\item Při oxidaci přechází bezbarvá forma (\ce{Cu^I}) na modrou (\ce{Cu^I^I}).
	\end{itemize}

	\begin{figure}
		\adjincludegraphics[height=.3\textheight]{img/Oxyhemocyanin.png}
	\end{figure}
	\end{column}
	\begin{column}{.4\textwidth}
		\begin{figure}
			\adjincludegraphics[width=\textwidth]{img/Hemocyanin2.jpg}
		\end{figure}
	\end{column}
	\end{columns}
	\vfill
}

\section{Nikl}
\frame{
	\frametitle{}
	\vfill
	\begin{itemize}
		\item Oproti železu a kobaltu je biologický význam niklu výrazně nižší.
		\item {[NiFe]} hydrogenáza je enzym katalyzující reverzibilní přeměnu molekulárního vodíku v některých prokaryotních organismech:\footnote[frame]{\href{https://doi.org/10.1039/C3RA22668A}{Fundamentals and electrochemical applications of [Ni–Fe]-uptake hydrogenases}}
		\item \ce{H2 <=> 2 H+ + 2 e-}
		\item Struktura enzymu obsahuje aktivní místo tvořené ionty Fe a Ni vázanými přes sulfidické můstky.
		\item Železo je stabilně v oxidačním stavu II, redoxních dějů se účastní nikl.
	\end{itemize}
	\begin{figure}
		\adjincludegraphics[height=.4\textheight]{img/Nickel_Iron_Hydrogenase_Active_Site.png}
	\end{figure}
	\vfill
}

\section{Zinek}
\frame{
	\frametitle{}
	\vfill
	\begin{itemize}
		\item Zinek patří mezi nejdůležitější kovy pro rostliny i živočichy.
		\item Lidské tělo obsahuje asi 2--4 g zinku, většinu ve formě enzymů.
		\item Zinek je Lewisovská kyselina, proto je z katalytického hlediska velice zajímavý.
		\item Také je velice flexibilní z hlediska koordinační geometrie, proto umožňuje rychlou změnu konformace enzymu.
		\item Zinek je součástí mnoha metalloenzymů, zapojuje se do homeostázy, imunitní odpovědi, apoptózy, stárnutí buněk a je také důležitým antioxidantem.
		\item Nedostatek zinku se projevuje mnoha symptomy:\footnote[frame]{\href{https://ods.od.nih.gov/factsheets/Zinc-HealthProfessional/}{Zinc}}
		\begin{itemize}
			\item lámavostí vlasů a nehtů
			\item suchou a popraskanou kůží
			\item zpomalením růstu u dětí
			\item šeroslepostí
			\item nechutenstvím
		\end{itemize}
	\end{itemize}
	\vfill
}

\section{Kadmium}
\frame{
	\frametitle{}
	\vfill
	\begin{itemize}
		\item Toxicita kadmia je dána tím, že kadmium vstupuje do metabolických drah zinku. Tím tyto dráhy narušuje.
		\item Otravu je možné potlačit podáváním zinku.
		\item Při inhalaci dochází primárně k poškození plic.
		\item Kadmium může také do těla vstupovat kůží.
		\item Velkým problémem při otravě kadmiem je dlouhý poločas jeho eliminace, takže může docházet k postupné akumulaci kadmia v organismu i při expozici nižším dávkám.
		\item Při projevu symptomů jsou následky otravy nevratné a dochází k postupnému zhoršování stavu.
		\item Kadmium může také podpořit rozvoj rakoviny plic a prostaty. Na druhou stranu, u některých nádorů mohou může kadmium působit jako supresivní látka.
	\end{itemize}
	\vfill
}

\section{Rtuť}
\frame{
	\frametitle{}
	\vfill
	\begin{itemize}
		\item Rtuť je toxická ve všech formách, jako kov i jako anorganické a organokovové sloučeniny \ce{Hg^{2+}} a \ce{Hg$_2^{2+}$}.\footnote[frame]{\href{https://www.wikiskripta.eu/w/Intoxikace\_rtut\%C3\%AD\_a\_jej\%C3\%ADmi\_slou\%C4\%8Deninami}{Intoxikace rtutí a jejími sloučeninami}}
		\item K intoxikaci může dojít jak vlivem přírodních jevů (ze zemské kůry se uvolňuje i více než 5 000 tun rtuti ročně), tak vlivem průmyslové činnosti (těžba zlata, elektrolytické procesy, apod.).
		\item Kvůli vysoké těkavosti jsou často vdechována páry rtuti, která pak prostupuje z plic do dalších orgánů (ledvin, CNS, červených krvinek).
		\item Vysoká mobilita rtuti v organismu je dána její rozpustností v tucích, což umožňuje transport přes buněčné membrány.
		\item Při chronické expozici dochází k poškozování CNS, které se projevuje třesavkou, emocionální nestabilitou a změnami chování. Dochází také k poškození ledvin a v případě těhotných žen i k poškození plodu.
	\end{itemize}
	\vfill
}

\frame{
	\frametitle{}
	\vfill
	\begin{itemize}
		\item Při otravě rtutí se využívají chelatační činidla, které umožní rychlé vyloučení rtuti močí. Jde např. o 2,3-disulfanylpropan-1-ol (dimerkaprol).
		\item Při nižší expozici se používá také dimethylcystein.
		\item Je také možné využít 2,3-disulfanyljantarovou kyselinu (DMSA).
	\end{itemize}
	\begin{figure}
		\adjincludegraphics[width=0.8\textwidth]{img/disulfanylpropanol.png}
	\end{figure}
	\vfill
}

\section{Olovo}
\frame{
	\frametitle{}
	\vfill
	\begin{itemize}
		\item Olovo je těžký kov, je toxický i v malých koncentracích a má jak akutní, tak i chronické účinky.\footnote[frame]{\href{https://dx.doi.org/10.1515/intox-2015-0009}{Lead toxicity: a review}}
		\item Toxické účinky lze vysvětlit vazbou olova na SH- skupiny enzymů, čímž dochází k jejich deaktivaci.\footnote[frame]{\href{http://web2.mendelu.cz/af_239_nanotech/J_Met_Nano/0314/pdf/jmn3-08.pdf}{Působení olova na živé organismy}}
		\item Toxicita olova je velkým problémem u dětí, u nichž může zpomalit duševní vývoj.
		\item Typickými příznaky otravy olovem jsou bledost obličeje a rtů, nechutenství, anémie.
		\item Koncentrace olova v životním prostředí se stanovuje pomocí AAS, MS nebo diferenční pulzní voltametrie.
		\item Průmyslová spotřeba olova je průběžně snižována, využívají se bezolovnaté pájky, hledají se bezolovnaté náhrady střeliva.
	\end{itemize}
	\vfill
}

\section{Arsen}
\frame{
	\frametitle{}
	\vfill
	\begin{itemize}
		\item Arsenité sloučeniny jsou toxičtější než arseničné.
		\item Atoxyl byl využíván při léčbě spavé nemoci.
		\item Některé organické sloučeniny arsenu byly dříve využívány při léčbě syfilidy.
		\item V současnosti se sloučeniny arsenu využívají při léčbě africké trypanosomiasy.
	\end{itemize}
	\begin{figure}
		\adjincludegraphics[width=.8\textwidth]{img/Atoxyl.png}
	\end{figure}
	\vfill
}

\section{Antimon}
\frame{
	\frametitle{}
	\vfill
	\begin{itemize}
		\item Kovový antimon neovlivňuje lidské zdraví.
		\item Oxid antimonitý a další nerozpustné antimonité sloučeniny jsou nebezpečné při vdechování.
		\item Otrava antimonitými sloučeninami je podobná otravě arsenikem.
		\item Oxid antimonitý je také potenciálně karcinogení.
	\end{itemize}
	\vfill
}

\section{Selen}
\frame{
	\frametitle{}
	\vfill
	\begin{columns}
		\begin{column}{.75\textwidth}
			\begin{itemize}
				\item Selen je větším množství toxický, ale ve stopovém množství je pro živočichy nezbytný.\footnote[frame]{\href{https://dx.doi.org/10.1001/archinternmed.2009.495}{Acute Selenium Toxicity Associated With a Dietary Supplement}}
				\item Je součástí aminokyselin selenocysteinu a selenomethioninu.
				\item Komerčně jsou dostupné doplňky stravy obsahující selen.\footnote[frame]{\href{https://www.bezpecnostpotravin.cz/selen-zdroje-ucinky-a-zasobovani.aspx}{Selen – zdroje, účinky a zásobování}}
				\item Doporučená denní dávka selenu pro člověka je 1~mg.kg$^{-1}$.\footnote[frame]{\href{https://www.wikiskripta.eu/w/Selen}{Selen}}. Dávky vyšší než 10~mg.kg$^{-1}$.den$^{-1}$ jsou toxické.
				\item Přirozeným zdrojem selenu jsou cereálie a mořské produkty.
				\item Otravy selenem jsou vzácné, akutní otrava se projevuje česnekovým zápachem potu a z úst (\ce{Se(CH3)2}). Chronická vypadáváním vlasů a nehtů.
			\end{itemize}
		\end{column}
		\begin{column}{.35\textwidth}
			\begin{figure}
				\adjincludegraphics[height=.75\textheight]{img/Se-AminoAcids.png}
			\end{figure}
		\end{column}
	\end{columns}
	\vfill
}

\frame{
	\frametitle{}
	\vfill
	\textbf{Obsah selenu v potravinách}
	\\
	\begin{tabular}{|l|l|}
		\hline
		\textbf{Potravina} & \textbf{Obsah selenu ($\mu$g/kg)} \\\hline
		Rostlinné oleje & méně než 5 \\\hline
		Ovoce & méně než 10 \\\hline
		Zelenina & 10--30 \\\hline
		Obiloviny & 10--500 \\\hline
		Houby & 20--100 \\\hline
		Hovězí maso & 20--80 \\\hline
		Drůbeží maso & 30--100 \\\hline
		Vepřové maso & 50--150 \\\hline
		Játra & 50--200 \\\hline
		Vejce & 100--200 \\\hline
		Ryby a měkkýši & 200--500 \\\hline
		Ledviny & 500--2000 \\\hline
		Para-ořechy, brazilské ořechy & 2000--5000 \\\hline
	\end{tabular}
	\vfill
}

\frame{
	\frametitle{}
	\vfill
	\begin{figure}
		\adjincludegraphics[height=0.2\textheight]{img/ebselen.png}
	\end{figure}
	\begin{itemize}
		\item Syntetické léčivo \textit{ebselen} má anti-oxidační účinky a zdá se být slibným léčivem proti COVID-19.\footnote[frame]{\href{https://www.nature.com/articles/s41586-020-2223-y}{Structure of Mpro from SARS-CoV-2 and discovery of its inhibitors}}
		\item Syntéza ebselenu a jeho derivátů probíhá podle schématu:\footnote[frame]{\href{https://doi.org/10.1002/hc.21164}{Synthesis and Antioxidant Activities of Novel Chiral Ebselen Analogues}}
	\end{itemize}
	\begin{figure}
		\adjincludegraphics[width=\textwidth]{img/Ebselen_Routes.png}
	\end{figure}
	\vfill
}

\section{Tellur}
\frame{
	\frametitle{}
	\vfill
	\begin{columns}
		\begin{column}{.55\textwidth}
			\begin{itemize}
				\item Tellur není příliš rozšířený v biologických systémech a jeho toxikologie není dosud příliš prozkoumaná.\footnote[frame]{\href{https://doi.org/10.1007/978-1-4614-1533-6_504}{Tellurium in Nature}}
				\item Některé houby (např. \textit{Aspergillus fumigatus} a \textit{Aspergillus terreus}) dokáží místo síry využívat tellur.\footnote[frame]{\href{https://doi.org/10.1007/BF02917437}{Incorporation of tellurium into amino acids and proteins in a tellurium-tolerant fungi}}
			\end{itemize}
		\end{column}
		\begin{column}{.5\textwidth}
			\begin{figure}
				\adjincludegraphics[width=\textwidth]{img/Aspergillus_on_tomato.jpg}
			\end{figure}
		\end{column}
	\end{columns}
	\vfill
}

\input{../Last}

\end{document}